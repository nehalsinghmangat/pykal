\documentclass[11pt]{article}
\usepackage[utf8]{inputenc}
\usepackage[T1]{fontenc}
\usepackage{graphicx}
\usepackage{longtable}
\usepackage{wrapfig}
\usepackage{rotating}
\usepackage[normalem]{ulem}
\usepackage{amsmath}
\usepackage{amssymb}
\usepackage{capt-of}
\usepackage{hyperref}
\usepackage{tikz-cd}
\usepackage{mathtools}
\usepackage{extpfeil} % gives \xtwoheadrightarrow and \xrightarrowtail
\author{Nehal Mangat}
\date{\today}
\title{Homework 6 -- MATH 744}

\setlength{\parindent}{0pt} % no paragraph indent
\setlength{\parskip}{1em}   % extra vertical space between paragraphs

\begin{document}

We define a small category \( \mathbf{FinSet} \) whose objects are finite discrete sets 
and whose morphisms are set functions. Likewise, let \( \mathbf{Dyn} \) denote the 
category of discrete-time dynamical systems, where an object is a pair 
\( (A,f) \) with \( f : A \rightarrow A \), and a morphism 
\( \phi : (A,f) \rightarrow (B,h) \) is a function \( \phi : A \rightarrow B \) satisfying
\[
\phi \circ f = h \circ \phi,
\]
so that \( \phi \) intertwines the dynamics.

Define a functor
\[
F : \mathbf{FinSet} \longrightarrow \mathbf{Dyn},
\]
that assigns to each finite set \( A \) the dynamical system \( (A,f) \) obtained by 
equipping \( A \) with an update rule \( f : A \rightarrow A \) (determined by the 
parameters \( P \subset \mathcal{U} \)). On morphisms, the functor acts by
\[
F(a) = f : A \rightarrow A,
\]
where the algorithm \( a : A \rightarrow B \) factors through a dynamical update 
followed by an output evaluation.

Similarly, define an output functor
\[
G : \mathbf{Dyn} \longrightarrow \mathbf{FinSet},
\]
that maps a dynamical system \( (A,f) \) to its output set \( B \), together with an 
output function \( g : A \rightarrow B \).

With these constructions in place, an algorithm 
\( a : A \rightarrow B \) admits the functorial factorization
\[
a = G(F(A)) = g \circ f,
\]
where \( f \) evolves the internal state and \( g \) evaluates the observable output. 
Thus every algorithm on a finite discrete set can be viewed as a discrete-time 
dynamical system together with an output map, and this decomposition is canonical 
with respect to the functorial structure above.
Denote by \( \mathcal{U} \) the universe of finite discrete sets and denote by \( A \subset \mathcal{U} \) a set upon which an action by a state machine is well-defined. Then we can define an algorithm \( a \) as a set function \( a: A \rightarrow B \) such that.

Recall that any finite discrete set is a 0-manifold and recall that a map on a manifold is a function \( f: A \rightarrow A \). Define the parameter space \( P \subset  \mathcal{U} \). Then the function \( f: A \rightarrow  A \) such that \( x_{k+1} = f(x,\,P) \) is a discrete time dynamical system. A smooth map \( g : A \rightarrow B \) is called an output function. Then there exists \( f \) adn \( g \) such that the composition \( a = g \circ f(a) \) for all \( a \in A \) (this is done functorially, as follows).


\end{document}
